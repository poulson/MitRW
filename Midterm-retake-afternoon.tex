\documentclass[11pt]{article}
\usepackage{amsmath,amsthm,amssymb}

\begin{document}
\noindent
{\bf Mathematics in the Real World: Math 16 / Stats 90}\\
{\bf Prof. Jack Poulson}\\
{\bf Midterm retake (afternoon): May 4, 2015}\\

\begin{enumerate}

\item
  Given the dataset 
  \[
    10.5, 7, 2, -5, -20, 11, 12, 27, 30,
  \]
  \begin{enumerate}
  \item
   (25 pts) Draw and label a box and whisker plot for the dataset.
  \item 
   (10 pts) Compute the best estimates of the dataset with respect to the 
            two-norm and max-norm.
  \end{enumerate}

\item 
  Consider a fair random walk:
  \begin{enumerate}
  \item
    (10 pts) Write down the expected change in squared distance from zero
             if the walk is currently at position $k$ and takes one more
             step.
  \item
    (10 pts) Write down the expected change in (absolute) distance from zero
             if the walk is currently at position $k \neq 0$ and takes one more
             random step, and compute the expected change from taking a step
             at $k = 0$.
  \item
    (10 pts) Describe the behaviour (as precisely as possible) of the Variance
             and Mean Absolute Deviation of the random walk as the number of 
             steps increases.
  \end{enumerate}

\item
  Consider a random walk where the first step is twice as likely to be backward
  as forward, but subsequent steps prefer to go in the {\bf opposite} direction
  as the preceding step by a factor of two.
  For example, if the first step was forward, the odds of the next step being
  backward are $2/3$. But, if the first step was forward and the second step 
  was backward, then the odds of the third step being forward are $2/3$.
  If the random walk is allowed to continue for {\bf three steps}:
  \begin{enumerate}
  \item (5 pts) List the possible final locations.
  \item (10 pts) Compute the likelihoods of each of these locations.
  \item (5 pts) Compute the expected location.
  \item (15 pts) Compute the variance , standard deviation, and 
                 Mean Absolute Deviation.
  \end{enumerate}
\end{enumerate}

\end{document}
