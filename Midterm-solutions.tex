\documentclass[11pt]{article}
\usepackage{amsmath,amsthm,amssymb}

\begin{document}
\noindent
{\bf Mathematics in the Real World: Math 16 / Stats 90}\\
{\bf Prof. Jack Poulson}\\
{\bf Midterm Solutions: May 1, 2015}\\

\begin{enumerate}

\item
  Given the dataset 
  \[
    1, 2, 10, 10.5, 11, 11.5, 12, 15, 20,
  \]
  \begin{enumerate}
  \item
   (25 pts) Draw and label a box and whisker plot for the dataset.
  \item 
   (10 pts) Compute the best two-norm and max-norm estimates.
  \end{enumerate}

\item 
  Suppose that you took $50$ random steps in a random walk where you were four 
  times as likely to step forwards as backwards in each step.
  \begin{enumerate}
  \item
    (10 pts) How many unique walks can end at $+20$? \\
    (Providing the formula is enough.)
  \item
    (10 pts) What is the probability of each of the paths that ends at $+20$? \\
    (Providing the formula is enough.)
  \item
    (10 pts) What is the probability of ending at $+20$ after $50$ steps? \\
    (Providing the formula is enough.)
  \end{enumerate}

\item
  Consider a random walk where the first step is equally-likely to be forward 
  as backward, but subsequent steps prefer to go in the opposite direction as 
  the preceding step by a factor of two.
  For example, if the first step was forward, the odds of the next step being
  backward are $2/3$. But, if the first step was forward and the second step 
  was backward, then the odds of the third step being forward are $2/3$.
  If the random walk is allowed to continue for {\bf three steps}:
  \begin{enumerate}
  \item (5 pts) List the possible final locations.
  \item (10 pts) Compute the likelihoods of each of these locations.
  \item (5 pts) Compute the expected location.
  \item (10 pts) Compute the variance and Mean Absolute Deviation.
  \item (5 pts) Compute the Interquartile Range.
  \end{enumerate}
\end{enumerate}

\end{document}
